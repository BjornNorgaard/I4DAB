\subsubsection{Operatorer Terminologi}

\begin{tabu}[h]{lX}
	\toprule
	\textbf{Navn} & \textbf{Beskrivelse}\\
	\midrule
	Relation & Et set af tupler\\
	Tuple & Er en samling af attributter\\
	Attribut & Er en kolonne\\
	Domæne & Er typen af data i en kolonne\\
	Set & Matematisk definition af en samling objekter i en relation, \textit{ingen} dubletter\\
	\bottomrule
\end{tabu}

%\paragraph{Relation}
%Er et set af tupler.
%
%\paragraph{Tuple}
%Er en samling af attributter.
%
%\paragraph{Attribut}
%Er en kolonne.
%
%\paragraph{Domæne}
%Er typen af data i en kolonne.
%
%\paragraph{Set}
%Er en matematisk definition af en samling af objekter i en relation, som \textit{ikke} indeholde dubletter.

\paragraph{Udtryk}

Et udtryk vil være på denne form:

\begin{equation*}
operator_{betingelse}(\text{i tabel})
\end{equation*}

%Til beskrivelse af diverse operatorer bruger vi tabel \ref{tab:stud}.
%
%\begin{table}[H]
%	\centering
%	\begin{tabular}{lrrl}
%		\toprule
%		\textbf{Navn}	&\textbf{Alder}	&\textbf{Gennemsnit}&\textbf{Email}\\
%		\midrule
%		John Derp		& 24 			& 7,7	& john@derp.me			\\			
%		Hans Hansen		& 45 			& 9,8	& hans@landmand.dk		\\			
%		Brian Jensen	& 22 			& 4,5	& brian@randers.dk		\\			
%		Signe Andersen	& 25 			& 8,6	& signe@hotmail.com		\\
%		\bottomrule
%	\end{tabular}
%	\caption{Tabel over studerende.}
%	\label{tab:stud}
%\end{table}