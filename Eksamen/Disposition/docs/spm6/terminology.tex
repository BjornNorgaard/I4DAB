\subsubsection{Terminologi}

\begin{table}[H]
	\begin{tabu}{lX}
		\toprule
		\textbf{Navn} & \textbf{Beskrivelse}\\
		\midrule
		Relation & Et set af tupler\\
		Tuple & Er en samling af attributter\\
		Attribut & Er en kolonne\\
		Domæne & Er typen af data i en kolonne\\
		Set & Matematisk definition af en samling objekter i en relation, \textit{ingen} dubletter\\
		Udtryk & $operator_{betingelse}(\text{tabel})$\\
		\bottomrule
	\end{tabu}
	\caption{Beskrivelser af termer indenfor Relationel Algebra.}
\end{table}

Der er ikke 'nøgler' i RA. Nøgler er er teknisk implementering, for at opfylde regler indenfor RA.

%Et udtryk vil være på denne form:
%
%\begin{equation*}
%operator_{betingelse}(\text{i tabel})
%\end{equation*}

%Til beskrivelse af diverse operatorer bruger vi tabel \ref{tab:stud}.
%
%\begin{table}[H]
%	\centering
%	\begin{tabular}{lrrl}
%		\toprule
%		\textbf{Navn}	&\textbf{Alder}	&\textbf{Gennemsnit}&\textbf{Email}\\
%		\midrule
%		John Derp		& 24 			& 7,7	& john@derp.me			\\			
%		Hans Hansen		& 45 			& 9,8	& hans@landmand.dk		\\			
%		Brian Jensen	& 22 			& 4,5	& brian@randers.dk		\\			
%		Signe Andersen	& 25 			& 8,6	& signe@hotmail.com		\\
%		\bottomrule
%	\end{tabular}
%	\caption{Tabel over studerende.}
%	\label{tab:stud}
%\end{table}