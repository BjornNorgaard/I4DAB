\subsection{Join $\Join$}

\url{https://en.wikipedia.org/wiki/Join_(SQL)#Cross_join}\\

Kombinerer tabeller i en relationel database. Returnere \textit{Cartesian Product}.\\
Vi har tabellen vist på figur~\ref{fig:employee_dept} og tilsvarende vises bruges af Join på listing~\ref{code:crossjoin}.

\begin{figure}[H]
\centering
\includegraphics[width=0.6\linewidth]{figs/spm6/employee_dept}
\caption{Eksempel til Join.}
\label{fig:employee_dept}
\end{figure}

Gældende notation ses herunder:

\begin{equation*}
Employee~JOIN_{betingelse} Department
\end{equation*}

\begin{lstlisting}[caption=SQL for Cross Join,label=code:crossjoin,morekeywords={SELECT, FROM, WHERE, CROSS, JOIN}]
// Example of an explicit cross join:
SELECT * FROM employee CROSS JOIN department;

// Example of an implicit cross join:
SELECT * FROM employee, department;
\end{lstlisting}

\subsubsection{Cartesian Product}
Binær operator. Også kaldet \textit{Cross Product} eller \textit{Cross Join}. 

Kombinere tupler i en relation med alle tupler i en anden relation.\\

Som det også kan ses i figur~\ref{fig:cartesian_product} så får hver attribut/kolonne i relation R sin \textit{egen udgave} af hver attribut/kolonne i relation S.

\begin{figure}[H]
	\centering
	\includegraphics[width=0.6\linewidth]{figs/spm6/cartesianproduct}
	\caption{Eksempel på \textit{cartesian product}.}
	\label{fig:cartesian_product}
\end{figure}

\subsubsection{Natural Join $\Join$}

\textit{''The result of the natural join is the set of all combinations of tuples in R and S that are equal on their common attribute names. For an example consider the tables Employee and Dept and their natural join:''}



\begin{figure}[H]
\centering
\includegraphics[width=\linewidth]{figs/spm6/naturaljoin}
\caption{Natural Join illustreret.}
\label{fig:naturaljoin}
\end{figure}

\begin{lstlisting}[caption=SQL for Cross Join,label=code:crossjoin,morekeywords={SELECT, FROM, WHERE, CROSS, JOIN, NATURAL}]
SELECT * FROM employee NATURAL JOIN department;
\end{lstlisting}


