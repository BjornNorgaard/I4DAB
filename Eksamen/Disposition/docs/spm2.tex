\section{Spørgsmål 2}

\subsection{Fokuspunkter}
\begin{itemize}
	\item Hvorledes omsættes et konceptuelt databasedesign til en konkret database?
	\item Kom bla. ind på begreberne "\textit{designregler}, \textit{relations modellering}, samt \textit{skema}.
\end{itemize}

\subsection{Litteratur}
\begin{itemize}
	\item Kap. 2 fra \textit{Database eLearning} - $http://db.grussell.org/index.html)$
	\begin{itemize}
		\item Database Analysis.
		\item Entity Relationship Modelling.
		\item Mapping ER Models into relations.
		\item Advanced ER Mapping.
	\end{itemize}
	\item Database Modelling and Design
	\begin{itemize}
		\item Kap. 1 (s. 1 - 11).
		\item Kap. 3 (s. 35 - 53).
		\item Kap. 5 (s. 85 - 108).
	\end{itemize}
\end{itemize}

\newpage

% must
\subsection{Hvorledes omsættes et konceptuelt databasedesign til en konkret database?}

% must
\subsection{De 10 Designregler}

\begin{enumerate}
	\item \textbf{Entity}\\
	Én entitet skal svare en én tabel. Entiteten skal have samme navn som tabellen. Skal også helst være et navneord i ental. 
	
	\item \textbf{Many-to-many binary relationship}\\
	Entiteternes primærnøgler bliver sammensat i en ny entitet, hvor de er fremmednøgler.
	
	\item \textbf{One-to-many binary relationship}\\
	''Mange''-entiteten skal have fremmednøglen, ellers brydes 1. normalform\todo{le fuq? Indsæt muligvis pageref til andet spm}.
	
	\item \textbf{Recursive binary relationship}\\
	Samme regler er gældende som ved binære relationer.
	
	\item \textbf{Ternary relationship}\\
	En weaktabel oprettes med en primær nøgle der er sammensat af de 3 fremmednøgler.\todo{Slide says ''+other stuff'', what other stuff?}
	
	\item \textbf{Attribute of an entity}\\
	En attribut af en entitet kan mappes direkte som en attribut i entitetens tabel.
	
	\item \textbf{Generalization super-class (super-type) entity}\\
	En super entitet mappes direkte til en SQL tabel.
	
	\item \textbf{Generalization super-class (subtype) entity}\\
	En sub entitet mappes til sin egen table for super entiteten. Primær nøglen fra super entiteten
	bliver fremmednøgle hos sub entiteten.
	
	\item \textbf{Mandatory constrint (1 lower bound) on the “one” side of a one-to-many relationship}\\
	Fremmednøglen skal ligge på mange siden, og sættes til ’not null’. Eks. En video skal ejes af en
	butik, butik behøver ikke eje en video. Ved at sætte fremmednøglen på mange siden får vi at; en
	butik kan godt oprettes uden en video, men en video skal tilknyttes en butik.
	
	\item \textbf{Subject gets targets Primary key as forreing key}\\
	I en 1 til 1 relation skal man identificere subject og target. Target har interesse i subjekt.
	
\end{enumerate}

% must
\subsection{Relationsmodellering}

% must
\subsection{Skema}