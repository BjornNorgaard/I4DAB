\section{Spørgsmål 8}

% Hvad er SQL kode afviklet på databasesiden? Herunder: hvorledes programmeres/erklæres og bruges Stored Procedure, Function, Trigger og View?

\subsection{Fokuspunkter}
\begin{itemize}
	\item Hvad er SQL kode afviklet på databasesiden?
	\begin{itemize}
		\item Hvorledes programmeres/erklæres og bruges Stored Procedure, Function, Trigger og View?
	\end{itemize}
\end{itemize}

\subsection{Litteratur}
\begin{itemize}
	
	\item Fra teori: Database Modeling and Design. Logical Design 5'th Ed.
	\begin{itemize}
		\item Appendix "The Basics of SQL" page 261-277.
	\end{itemize}
	
%	\item Fra Database eLearning: \url{http://db.grussell.org/index.html}.
%	\begin{itemize}
%		\item Database Analysis and ER Modelling.
%		\begin{itemize}
%			\item Database Analysis.
%			\item Entity Relationship Modelling - 2.
%			\item Advanced ER Mapping.
%		\end{itemize}
%	\end{itemize}
	
	\item Fra wikipedia:
	\begin{itemize}
		\item \href{https://en.wikipedia.org/wiki/View_(SQL)}{View (database)}.
		\item \href{https://en.wikipedia.org/wiki/Stored_procedure}{Stored procedure}.
		\item \href{https://en.wikipedia.org/wiki/Database_trigger}{Database trigger}.
	\end{itemize}
	
%	\item Fra Agile Data Home Page:
%	\begin{itemize}
%		\item 
%	\end{itemize}
\end{itemize}

\newpage

% must
\subsection{Hvad er SQL kode afviklet på databasesiden?}
%Når man kommunikerer med en SQL database, gøres det via nogle specifikke statements, som databasen forstår (CREATE;READ;UPDATE;DELETE). Disse statements bruges til kommunikation, og gemte funktioner/procedure på DB.

SQL	kode gemt på databasen kan være	med	til	at skabe et	fælles interface som	brugere	af databasen kan anvende.

\begin{itemize}
	\item SQL koden	på databasesiden er	ikke afhængig af platformen	som	tilgår	databasen, eller hvilket sprog den platform	bruger.
	\item Godt hvis forskellige klienter skal tilgå databasen og udføre de samme operationer.
	\item Enkapsulerer business logik ét sted.
	\item SQL kode der er gemt på databasen	har	mulighed for allerede være omsat til	RA og maskinkode.
	\item Undgår at	sende lange	SQL	data til databasen,	man	kan	blot kalde en	funktion.
\end{itemize}

% must
\subsubsection{Hvorledes programmeres/erklæres og bruges Stored Procedure, Function, Trigger og View?}

% must
\subsubsection{Stored Procedure}
% Hvorledes programmeres/erklæres og bruges...

\begin{itemize}
	\item Er et gemt statement.
	\item Returnerer et resultatsæt (relation).\todo{Forskel på en relation og en tabel?}
\end{itemize}

% must
\subsubsection{Function}
% Hvorledes programmeres/erklæres og bruges...

\begin{itemize}
	\item Returnerer en værdi (eller en tabel).
	\item Der findes 3 typer:
	
	\begin{itemize}
		\item \textbf{Scalar} - returnerer en enkelt værdi, med eller uden inddata\todo{Woot is inddata? En fejl?}.
		\item \textbf{Inline table} - returnerer en tabel med eller uden inddata.
		\item \textbf{Multi-statement} - returnerer en ny tabel.\todo{Hvad er forskellen på Inline og multi?}
	\end{itemize}
\end{itemize}

% must
\subsubsection{Trigger}
% Hvorledes programmeres/erklæres og bruges...

\begin{itemize}
	\item Minder om C\# events.
	\item Kan bruges til at logge i en fil. % mindre ting
	\item En trigger kan kun behandler et sql statement med én value.
\end{itemize}

% must
\subsubsection{View}
% Hvorledes programmeres/erklæres og bruges...
 Fordele ved et View er at fordi det er \textbf{virtuelt}, så kan brugere få adgang til dataen, uden at redigere i det. I et view kan der også udføres beregninger, såsom at summere eller beregne gennemsnit, men eftersom det \textbf{ikke påvirker dataen} så ødelægges ikke noget.

\begin{itemize}
	\item Er en virtuel tabel, baseret på resultatet af et SQL statement.
	\item Indeholder rækker og kolonner, som er fra en eller flere tabeller i databasen.
	\item Kan anvendes med SQL statements (WHERE og JOIN), som om det hele kom fra én tabel.
	%\item Kan være READ-ONLY, eller EDITABLE
\end{itemize}

\paragraph{Oprettelse af view} her følger kodeeksemplet for oprettelse og anvendelse et view. \\

\textit{''The view "Current Product List" lists all active products (products that are not discontinued) from the "Products" table. The view is created with the following SQL:''}

\begin{lstlisting}[caption=Kodeeksempel for oprettelse af View,label=code:view,morekeywords={CREATE, VIEW, AS, SELECT, FROM, WHERE}]
// Oprettelse af view
CREATE VIEW [Current Product List] AS
SELECT 	ProductID,ProductName
FROM 	Products
WHERE 	Discontinued=No
\end{lstlisting}

Da vi view'et bruges som vist her: 

\begin{lstlisting}[caption=Kodeeksempel for anvendelse af View,label=code:view]
// Anvendelse af view
SELECT * FROM [Current Product List]
\end{lstlisting}

Et andet eksempel:\\

\textit{''Another view in the Northwind sample database selects every product in the "Products" table with a unit price higher than the average unit price:''}

\begin{lstlisting}[caption=Et andet kodeeksempel for anvendelse af View,label=code:view]
// Anvendelse af view
CREATE VIEW [Products Above Average Price] AS
SELECT 	ProductName,UnitPrice
FROM 	Products
WHERE 	UnitPrice>(SELECT AVG(UnitPrice) FROM Products)
\end{lstlisting}

\begin{lstlisting}[caption=Kodeeksempel for anvendelse af View,label=code:view]
// Anvendelse af view
SELECT * FROM [Products Above Average Price]
\end{lstlisting}

\paragraph{Opdatering/Ændring af view} syntax for ændring af et view:

\begin{lstlisting}[caption=Opdatering af view]
// How to opdate view
CREATE OR REPLACE VIEW view_name AS
SELECT 	column_name(s)
FROM 	table_name
WHERE 	condition
\end{lstlisting}

\begin{lstlisting}[caption=Andet eksempel på opdatering af et view.]
// How to opdate view ex 2
CREATE OR REPLACE VIEW [Current Product List] AS
SELECT 	ProductID,ProductName,Category
FROM 	Products
WHERE 	Discontinued=No
\end{lstlisting}

\paragraph{Drop/slet view} hvordan kan sletter/dropper et view: 

\begin{lstlisting}[caption=Dropping af view]
// How to drop a view
DROP VIEW view_name
\end{lstlisting}